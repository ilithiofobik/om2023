\documentclass{article}

% Language setting
% Replace `english' with e.g. `spanish' to change the document language
\usepackage[polish]{babel}
\usepackage[utf8]{inputenc}
\usepackage{polski}
% Set page size and margins
% Replace `letterpaper' with `a4paper' for UK/EU standard size
\usepackage[letterpaper,top=2cm,bottom=2cm,left=3cm,right=3cm,marginparwidth=1.75cm]{geometry}

% Useful packages
\usepackage{amsmath}
\usepackage{amsfonts}
\usepackage{graphicx}
\usepackage[colorlinks=true, allcolors=blue]{hyperref}
%\usepackage{tikz-cd}

\title{Metody Optymalizacji - Laboratorium 2}
\author{Wojciech Sęk}

\def\XX{\mathcal{X}}
\def\N{\mathbb{N}}
\def\x{\textbf{x}}
\def\z{\textbf{z}}
\def\0{\textbf{0}}
\def\1{\textbf{1}}
\def\c{\textbf{c}}
\def\b{\textbf{b}}

\begin{document}
\maketitle
\section{Zadanie 1}
\subsection{Model}

\subsubsection{Zmienne decyzyjne}
Dla $m$ danych sposobów pocięcia standardowej deski na deski żądanych szerokości przez $x_j$ dla $j\in[m]$ oznaczamy liczbę desek pociętych w $j$-ty sposób.

\subsubsection{Ograniczenia}
\begin{itemize}
\item Niech $\alpha_{j,i}$ oznacza liczbę desek rodzaju $i$ wyciętych, gdy używamy $j$-tej metody cięcia deski. Niech $\delta_i$ oznacza podaż na deski rodzaju $i$. Suma wyprodukowanych desek danego rodzaju musi być równa podaży na dany rodzaj:
$$(\forall i \in [n])\left(\sum_{j=1}^m x_j \cdot \alpha_{j,i} = \delta_i\right)$$
\end{itemize}

\subsection{Funkcja celu}
Minimalizujemy sumę odpadów z produkcji wszystkich desek. Niech $\lambda_i$ oznacza szerokość odpadu w $i$-tej metodzie. Funkcją celu jest wtedy
$$f(\x)=\sum_{i=1}^m x_i \cdot \lambda_{i}$$
\subsection{Wyniki}
Dla standardowej deski szerokości 22 i żądania na 80 desek szerokości 3, 120 desek szerokości 5 i 110 desek szerokości 7 optymalnie będzie rozciąć deski wg następującej tabeli, gdzie $i$ oznacza numer metody (łącznie metod było 42, pomijamy wiersze dla $x_i=0$). Niech $\alpha_i(j)$ oznacza liczbę desek szerokości $i$ uzyskanych w danej metodzie:
\begin{center}
\begin{tabular}{|c|c|c|c|c|c|}
\hline 
$i$ & $x_i$ & $x_i\cdot \alpha_i(3)$ & $x_i\cdot\alpha_i(5)$ & $x_i\cdot\alpha_i(7)$ & $x_i\cdot\lambda_i$ \\
\hline
26 & 11 & 0 & 0 & 33 & 11 \\
\hline
33 & 9 & 45 & 0 & 9 & 0 \\
\hline
37 & 33 & 0 & 99 & 33 & 0 \\
\hline
39 & 14 & 14 & 14 & 28 & 0 \\
\hline
42 & 7 & 21 & 7 & 7 & 7 \\
\hline
\hline
$\Sigma_i$ & 11 & 80 & 120 & 110 & 18 \\
\hline
\end{tabular}
\end{center}

Mamy wtedy resztki szerokości 18.

\section{Zadanie 2}
\subsection{Model}
\subsubsection{Zmienne decyzyjne}
Dla $n$ zadań i horyzontu czasowego $T$ wprowadzamy binarną zmienną $x_{i,j}$, która dla $i\in[n],j\in[T]$ oznacza, że zadanie $i$ rozpoczyna się w momencie $j-1$. $T$ to najpóźniejszy możliwy czas rozpoczęcia zadania, który równa się
$$\max_{i\in [n]} r_i + \sum_{i\in [n]} p_i + 1$$ 
gdzie $r_i$ to najwcześniejszy moment rozpoczęcia zadania $i$ a $p_i$ to czas jego trwania. W najgorszym przypadku zadania rozpoczną się z największym opóźnieniem $r$, muszą trwać co najmniej tyle co suma ich wykonywania. Czynnik $+1$ wynika z tego, że $x_{*,t}$ oznacza moment $t-1$.  


\subsubsection{Ograniczenia}
\begin{itemize}
\item Każde zadanie rozpoczyna się dokładnie raz:
$$(\forall i \in [n])\left(\sum_{t\in [T]}x_{j,t}=1\right)$$
\item $j$-te  zadanie rozpoczyna się nie wcześniej niż w momencie $r_j$
$$(\forall j \in [n])\left(\sum_{t\in [T]}(t-1)\cdot x_{j,t}\geq r_j\right)$$
\item W dowolnym momencie czasu wykonujemy co najwyżej jedno zadanie
$$(\forall t \in [T])\left(\sum_{j\in n} \left(\sum_{s\in \max(1, t+1-p_j)}^t x_{j,s} \right) \leq 1\right)$$
\end{itemize}

\subsection{Funkcja celu}
Niech $w_j$ oznacza wagę $j$-ego zadania. Minimalizujemy ważoną sumę czasów zakończenia zadań
$$f(\x)=\sum_{j \in [n]}\sum_{t\in[T]} w_j * (t-1+p_j) * x_{j,t}$$


\subsection{Wyniki}
W ramach sprawdzenia implementacji użyłem dwóch zadań o następujących parametrach
$$p_1=10,\quad p_2=10,\quad  w_1=10000,\quad  p_2=1,\quad  r_1=5,\quad  r_2=0$$
Model określił, że zadanie pierwsze powinno rozpocząć się w momencie 5 a drugie w momencie 15. Jest to zgodne z oczekiwaniami, że mimo większego ograniczenia $r$ na zadanie pierwsze, zaczynamy od niego, ponieważ ma dużo większą wagę.


\section{Zadanie 3}
\subsection{Model}
\subsubsection{Zmienne decyzyjne}
Dla $n$ zadań, $m$ maszyn i horyzontu czasowego $T$ wprowadzamy binarną zmienną $x_{i,j,k}$, która dla $i\in[n],j\in[T],k\in[m]$ oznacza, że zadanie $i$ rozpoczyna się w momencie $j-1$ na maszynie $k$. $T$ to najpóźniejszy możliwy czas rozpoczęcia zadania, który równa się
$$\sum_{i\in [n]} p_i + 1$$ 
gdzie $p_i$ to czas trwania zadania $i$. W najgorszym przypadku zadania muszą trwać co najmniej tyle co suma ich wykonywania. Czynnik $+1$ wynika z tego, że $x_{*,t,*}$ oznacza moment $t-1$.\\
Wprowadzamy również zmienną $c_{MAX}$, która ma ograniczyć od góry czas zakończenia dowolnego zadania.

\subsubsection{Ograniczenia}
\begin{itemize}
\item Każde zadanie kończy najpóźniej w momencie $c_{MAX}$:
$$(\forall j \in [n])(\forall t \in [T])(\forall k \in [m])\left((t-1+p_j)\cdot x_{j,t,k} \leq c_{MAX }\right)$$
\item Każde zadanie rozpoczyna się dokładnie raz na dokładnie jednej maszynie:
$$(\forall i \in [n])\left(\sum_{t\in [T]}\sum_{k\in[m]}x_{j,t,k}=1\right)$$
\item W dowolnym momencie czasu na jednej maszynie wykonujemy co najwyżej jedno zadanie
$$(\forall t \in [T])(\forall k\in[m])\left(\sum_{j\in n} \left(\sum_{s\in \max(1, t+1-p_j)}^t x_{j,s,k} \right) \leq 1\right)$$
\item Niech $\pi_i$ oznacza zbiór poprzedników zadania $i$, gdzie przez poprzednika rozumiemy zadanie, którego czas zakończenia musi być mniejszy bądź równy czasowi rozpoczęcia zadania $i$. Warunek można zapisać jako:
$$(\forall b \in [n])(\forall a \in \pi_b)\left( \sum_{t=1}^{T-p_a+1}\sum_{k\in[m]} (t+p_a-1) \cdot x_{a,t,k}  \leq \sum_{t=1}^{T-p_b+1}\sum_{k\in[m]} (t-1) \cdot x_{b,t,k}  \right)$$
\end{itemize}

\subsection{Funkcja celu}
Minimalizujemy czas zakończenia ostatniego zadania
$$f(\x,c_{MAX})=c_{MAX}$$
 
\subsection{Wyniki}
Dla przykładowych danych zaplanowano następujący schemat o czasie wykonywania 9:
\begin{center}
\begin{tabular}{|c|c|c|c|}
\hline
\textbf{Przedział czasu} & \textbf{$M1$} & \textbf{$M2$} & \textbf{$M3$}\\
\hline
(0,1) & 2 & 3 &  \\
\hline
(1,2) & 2 & & 1 \\
\hline
(2,3) &  & 5 & 4 \\
\hline
(3,4) &  & 8 & 4 \\
\hline
(4,5) &  & 8 & 7 \\
\hline
(5,6) & 6 & 8 & 7 \\
\hline
(6,7) &  & 8 & 7 \\
\hline
(7,9) &  & 8 & 9 \\
\hline

\end{tabular}
\end{center}



\section{Zadanie 4}
\subsection{Model}
\subsubsection{Zmienne decyzyjne}
Dla $n$ zadań, i horyzontu czasowego $T$ wprowadzamy binarną zmienną $x_{i,j}$, która dla $i\in[n],j\in[T]$ oznacza, że zadanie $i$ rozpoczyna się w momencie $j-1$. $T$ to najpóźniejszy możliwy czas rozpoczęcia zadania, który równa się
$$\sum_{i\in [n]} \tau_i + 1$$ 
gdzie $\tau_i$ to czas trwania zadania $i$. W najgorszym przypadku zadania muszą trwać co najmniej tyle co suma ich wykonywania. Czynnik $+1$ wynika z tego, że $x_{*,t}$ oznacza moment $t-1$.\\
Wprowadzamy również zmienną $c_{MAX}$, która ma ograniczyć od góry czas zakończenia dowolnego zadania.

\subsubsection{Ograniczenia}

\begin{itemize}
\item Każde zadanie kończy najpóźniej w momencie $c_{MAX}$:
$$(\forall j \in [n])(\forall t \in [T])\left((t-1+\tau_j)\cdot x_{j,t} \leq c_{MAX }\right)$$
\item Każde zadanie rozpoczyna się dokładnie raz:
$$(\forall i \in [n])\left(\sum_{t\in [T]}x_{j,t}=1\right)$$
\item W dowolnym momencie dla każdego zasobu nie możemy przekroczyć jego chwilowego zużycia.  Niech $p$ to liczba zasobów, $N_i$ oznacza ilość zasobu $i$ oraz niech $r_{j,i}$ oznacza zapotrzebowanie zadania $j$ na zasób $i$. Wtedy warunek możemy określić jako:
$$(\forall t\in[T])(i \in [p])\left( \sum_{j=1}^n  \sum_{s=\max(1,t+1-\tau_j)}^t x_{j,s}\cdot r_{j,i} \leq N_i \right)$$
\item Niech $\pi_i$ oznacza zbiór poprzedników zadania $i$, gdzie przez poprzednika rozumiemy zadanie, którego czas zakończenia musi być mniejszy bądź równy czasowi rozpoczęcia zadania $i$. Warunek można zapisać jako:
$$(\forall b \in [n])(\forall a \in \pi_b)\left( \sum_{t=1}^{T-\tau_a+1} (t+\tau_a-1) \cdot x_{a,t}  \leq \sum_{t=1}^{T-\tau_b+1} (t-1) \cdot x_{b,t}  \right)$$
\end{itemize}

\subsection{Funkcja celu}
Minimalizujemy czas zakończenia ostatniego zadania
$$f(\x,c_{MAX})=c_{MAX}$$
 

\subsection{Wyniki}
Dla przykładowych danych znaleziono następujący schemat o czasie wykonywania 237:
\begin{center}
\begin{tabular}{|c|c|c|}
\hline
\textbf{Przedział czasu} & \textbf{Zużycie zasobu $r_1$} & \textbf{Wykonywane zadania}\\
\hline
(0,50) & 9 & 1 \\
\hline
(50,54) & 4 & 4 \\
\hline
(54,96) & 15 & 3,4 \\
\hline
(96,109) & 28 & 2,3 \\
\hline
(109,116) & 17 & 2 \\
\hline
(109,143) & 24 & 2,6 \\
\hline
(143,144) & 20 & 5,6 \\
\hline
(144,159) & 27 & 5,6,7 \\
\hline
(159,173) & 20 & 5,6 \\
\hline
(173,175) & 13 & 5 \\
\hline
(173,237) & 17 & 8 \\
\hline
\end{tabular}
\end{center}

\end{document}
