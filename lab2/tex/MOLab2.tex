\documentclass{article}

% Language setting
% Replace `english' with e.g. `spanish' to change the document language
\usepackage[polish]{babel}
\usepackage[utf8]{inputenc}
\usepackage{polski}
% Set page size and margins
% Replace `letterpaper' with `a4paper' for UK/EU standard size
\usepackage[letterpaper,top=2cm,bottom=2cm,left=3cm,right=3cm,marginparwidth=1.75cm]{geometry}

% Useful packages
\usepackage{amsmath}
\usepackage{amsfonts}
\usepackage{graphicx}
\usepackage[colorlinks=true, allcolors=blue]{hyperref}
%\usepackage{tikz-cd}

\title{Metody Optymalizacji - Laboratorium 2}
\author{Wojciech Sęk}

\def\XX{\mathcal{X}}
\def\N{\mathbb{N}}
\def\x{\textbf{x}}
\def\z{\textbf{z}}
\def\0{\textbf{0}}
\def\1{\textbf{1}}
\def\c{\textbf{c}}
\def\b{\textbf{b}}

\begin{document}
\maketitle
\section{Zadanie 1}
\subsection{Model}

\subsubsection{Zmienne decyzyjne}
Dla $m$ danych sposobów pocięcia standardowej deski na deski żądanych szerokości przez $x_j$ dla $j\in[m]$ oznaczamy liczbę desek pociętych w $j$-ty sposób.

\subsubsection{Ograniczenia}
\begin{itemize}
\item Niech $\alpha_{j,i}$ oznacza liczbę desek rodzaju $i$ wyciętych, gdy używamy $j$-tej metody cięcia deski. Niech $\delta_i$ oznacza podaż na deski rodzaju $i$. Suma wyprodukowanych desek danego rodzaju musi być równa podaży na dany rodzaj:
$$(\forall i \in [n])\left(\sum_{j=1}^m x_j \cdot \alpha_{j,i} = \delta_i\right)$$
\end{itemize}

\subsection{Funkcja celu}
Minimalizujemy sumę odpadów z produkcji wszystkich desek. Niech $\lambda_i$ oznacza szerokość odpadu w $i$-tej metodzie. Funkcją celu jest wtedy
$$f(\x)=\sum_{i=1}^m x_i \cdot \lambda_{i}$$
\subsection{Wyniki}
Dla standardowej deski szerokości 22 i żądania na 80 desek szerokości 3, 120 desek szerokości 5 i 110 desek szerokości 7 optymalnie będzie rozciąć deski wg następującej tabeli, gdzie $i$ oznacza numer metody (łącznie metod było 42, pomijamy wiersze dla $x_i=0$). Niech $\alpha_i(j)$ oznacza liczbę desek szerokości $i$ uzyskanych w danej metodzie:
\begin{center}
\begin{tabular}{|c|c|c|c|c|c|}
\hline 
$i$ & $x_i$ & $x_i\cdot \alpha_i(3)$ & $x_i\cdot\alpha_i(5)$ & $x_i\cdot\alpha_i(7)$ & $x_i\cdot\lambda_i$ \\
\hline
26 & 11 & 0 & 0 & 33 & 11 \\
\hline
33 & 9 & 45 & 0 & 9 & 0 \\
\hline
37 & 33 & 0 & 99 & 33 & 0 \\
\hline
39 & 14 & 14 & 14 & 28 & 0 \\
\hline
42 & 7 & 21 & 7 & 7 & 7 \\
\hline
\hline
$\Sigma_i$ & 11 & 80 & 120 & 110 & 18 \\
\hline
\end{tabular}
\end{center}

Mamy wtedy resztki szerokości 18.

\section{Zadanie 2}
\subsection{Model}
\subsubsection{Zmienne decyzyjne}
Dla $n$ zadań i horyzontu czasowego $T$ wprowadzamy binarną zmienną $x_{i,j}$, która dla $i\in[n],j\in[T]$ oznacza, że zadanie $i$ rozpoczyna się w momencie $j-1$. $T$ to najpóźniejszy możliwy czas rozpoczęcia zadania, który równa się
$$\max_{i\in [n]} r_i + \sum_{i\in [n]} p_i + 1$$ 
gdzie $r_i$ to najwcześniejszy moment rozpoczęcia zadania $i$ a $p_i$ to czas jego trwania. W najgorszym przypadku zadania rozpoczną się z największym opóźnieniem $r$, muszą trwać co najmniej tyle co suma ich wykonywania. Czynnik $+1$ wynika z tego, że $x_{*,t}$ oznacza moment $t-1$.  


\subsubsection{Ograniczenia}
\begin{itemize}
\item Każde zadanie rozpoczyna się dokładnie raz:
$$(\forall i \in [n])\left(\sum_{t\in T}x_{j,t}=1\right)$$
\item $j$-te  zadanie rozpoczyna się nie wcześniej niż w momencie $r_j$
$$(\forall j \in [n])\left(\sum_{t\in T}(t-1)\cdot x_{j,t}\geq r_j\right)$$
\item W dowolnym momencie czasu wykonujemy co najwyżej jedno zadanie
$$(\forall t \in [T])\left(\sum_{j\in n} \left(\sum_{s\in \max(1, t+1-p_j)}^t x_{j,s} \right) \leq 1\right)$$
\end{itemize}

\subsection{Funkcja celu}
Niech $w_j$ oznacza wagę $j$-ego zadania. Minimalizujemy ważoną sumę czasów zakończenia zadań
$$f(\x)=\sum_{j \in [n]}\sum_{t\in[T]} w_j * (t-1+p_j) * x_{j,t}$$


\subsection{Wyniki}
W ramach sprawdzenia implementacji użyłem dwóch zadań o następujących parametrach
$$p_1=10,\quad p_2=10,\quad  w_1=10000,\quad  p_2=1,\quad  r_1=5,\quad  r_2=0$$
Model określił, że zadanie pierwsze powinno rozpocząć się w momencie 5 a drugie w momencie 15. Jest to zgodne z oczekiwaniami, że mimo większego ograniczenia $r$ na zadanie pierwsze, zaczynamy od niego, ponieważ ma dużo większą wagę.


\section{Zadanie 3}
\subsection{Model}
\subsubsection{Zmienne decyzyjne}
Dla $n$ zadań, $m$ maszyn i horyzontu czasowego $T$ wprowadzamy binarną zmienną $x_{i,j,k}$, która dla $i\in[n],j\in[T],k\in[m]$ oznacza, że zadanie $i$ rozpoczyna się w momencie $j-1$ na maszynie $k$. $T$ to najpóźniejszy możliwy czas rozpoczęcia zadania, który równa się
$$\sum_{i\in [n]} p_i + 1$$ 
gdzie $p_i$ to czas trwania zadania $i$. W najgorszym przypadku zadania muszą trwać co najmniej tyle co suma ich wykonywania. Czynnik $+1$ wynika z tego, że $x_{*,t,*}$ oznacza moment $t-1$.  



\subsubsection{Ograniczenia}
Wprowadźmy ogólne oznaczenia parametrów:
\begin{itemize}
    \item $\varepsilon_{p, c}$ - współczynnik uzyskiwania z ropy $c\in C$  produktu $p\in P_C$ (w treści zadania $P_C=\{benzyna, olej, destylat, resztki\}$) 
    \item $\chi_{d}$ - współczynnik uzyskiwania produktu $d\in D_P$ w procesie krakowania destylatu (w treści zadania $D_P=\{benzyna, olej, resztki\}$) 
    \item $\sigma_{c}$ - udział siarki w oleju z ropy $c\in C$
    \item $\eta_c$     - udział siarki w oleju z krakowania destylatu z ropy $c\in C$
    
\end{itemize}
Ograniczenia właściwe:
\begin{itemize}
    \item Suma oleju wyprodukowanego z danego typu ropy musi równać się sumie ton tego oleju wykorzystanych do różnych celów:
    $$\left(\forall c \in C\right) \left(\varepsilon_{olej, c} \cdot  x_c = \sum_{o \in O_U} y_{o, c}\right)$$
    
    \item Suma destylatu wyprodukowanego z danego typu ropy musi równać się sumie ton tego destylatu wykorzystanych do różnych celów:
    $$\left(\forall c \in C\right) \left( \varepsilon_{destylat, c} \cdot  x_{c} = \sum_{d \in D_U} z_{d, c}\right)$$
    
    \item Niech $MIN_{silnikowe}$ oznacza minimalną liczbę ton paliw silnikowych do wyprodukowania. W całości procesu na paliwa silnikowe składa się benzyna ze wszystkich rodzajów ropy i z krakowania:
    $$\sum_{c \in C} \left(\chi_{benzyna} \cdot  z_{krak, c} + \varepsilon_{benzyna, c} \cdot  x_{c}\right) \geq MIN_{silnikowe}$$

    \item Niech $MIN_{domowe}$ oznacza minimalną liczbę ton domowych paliw olejowych do wyprodukowania. W całości procesu na domowe paliwa olejowe składa się olej z destylacji każdego rodzaju ropy wykorzystany do paliw domowych oraz olej z procesu krakowania destylatu:
    $$\sum_{c \in C} (\chi_{olej} \cdot  z_{krak, c} + y_{domowe, c}) \geq MIN_{domowe}$$

    \item Niech $MIN_{ciezkie}$ oznacza minimalną liczbę ton ciężkich paliw olejowych do wyprodukowania. W całości procesu na ciężkie paliwa olejowe składa się olej oraz destylat z destylacji każdego rodzaju ropy wykorzystane do paliw ciężkich, resztki z każdego etapu produkcji:
    $$\sum {c \in C} (y_{ciezkie, c} + z_{ciezkie, c} + \varepsilon_{resztki, c} \cdot  x_{c} + \chi_{resztki} \cdot  z_{krak, c})  \geq MIN_{ciezkie}$$

    \item Niech $MAX_S$ oznacza maksymalny udział siarki w domowych paliwach olejowych. Zatem ten udział przemnożony przez całość domowych paliw olejowych musi być większy lub równy od prawdziwego udziału siarki w poszczególnych produktach składających się na domowe paliwa olejowe, czyli olejach z destylacji ropy oraz olejach z krakowania:
    $$\sum_{c \in C} (\sigma_{c} \cdot  y_{domowe, c} + (\eta_{c} \cdot  \chi_{olej} \cdot  z_{krak, c})) \leq MAX_S \cdot  \sum_{c \in C} (\chi_{olej} \cdot  z_{krak, c} + y_{domowe, c})$$
\end{itemize}

\subsection{Funkcja celu}
Chcemy minimalizować koszt wytworzenia wszystkich paliw. Niech $\gamma_c$ oznacza koszt tony ropy typu $c$, $\delta$ oznacza koszt destylacji tony ropy a $\kappa$ oznacza koszt krakowania tony destylatu. Wtedy całkowity koszt wynosi:
$$f(\x, \z)=  \sum_{c \in C} ((\gamma_{c} + \delta) \cdot  x_{c} + \kappa \cdot  z_{krak, c})$$

\subsection{Wyniki}
Optymalnym rozwiązaniem będzie zakup wyłącznie tańszej ropy, czyli ropy typu $B1$. Ostatecznie:
\begin{itemize}
    \item Kupujemy 1026030.36876356 ton ropy $B1$
    \item 381561.822125814 ton oleju z destylacji ropy dodajemy do domowych paliw olejowych
    \item 28850.3253796095 ton oleju z destylacji ropy dodajemy do ciężkich paliw olejowych
    \item 92190.8893709328 ton destylatu z destylacji ropy dajemy do krakowania
    \item 61713.6659436009 ton destylatu z destylacji ropy dajemy do ciężkich paliw olejowych
\end{itemize}
Całkowity koszt wynosi 1345943600.86768 \$.


\section{Zadanie 4}
\subsection{Model}
\subsubsection{Zmienne decyzyjne}
Podejmujemy decyzje o ćwiczeniach oraz o treningach:
\begin{itemize}
    \item Przez $x_{g, c}=1$ oznaczamy, że bierzemy zajęcia z kursu $c\in C$ w grupie $g\in G$. W przeciwnym przypadku $x_{g, c}=0$. W treści zadania $C=\{algebra ,analiza, fizyka, chemia_min ,chemia_org\}$ i $G=\{I,II,III,IV\}$. 
    \item Przez $y_{p}=1$ oznaczamy, że trenujemy w grupie treningowej $p\in P$. W przeciwnym przypadku $y_p=0$. W treści zadania nie oznaczono tego zbioru, ale można zdefiniować $P_G=\{I,II,II\}$, które oznaczają kolejno treningi w pon 11-13 oraz w środę 11-13 i 13-15. 
\end{itemize}

\subsubsection{Ograniczenia}
Wprowadźmy ogólne oznaczenia parametrów:
\begin{itemize}
    \item $\sigma_{g,c}$ - oznacza godzinę rozpoczęcia zajęć z kursu $c\in C$ w grupie $g\in G$
    \item $\varepsilon_{g,c}$ - oznacza godzinę zakończenia zajęć z kursu $c\in C$ w grupie $g\in G$
    \item $\delta_{g,c}\in [1,2,3,4,5]$ - oznacza dzień zajęć z kursu $c\in C$ w grupie $g\in G$

    \item $\pi_{g,c}$ - oznacza punkty preferencji danych zajęć $c$ w grupie $G$
    

    \item $\sigma^{PE}_{p}$ - oznacza godzinę rozpoczęcia treningu $p$
    \item $\varepsilon^{PE}_{p}$ - oznacza godzinę zakończenia treningu $p$
    \item $\delta^{PE}_{p}\in [1,2,3,4,5]$ - oznacza dzień treningu $p$

    
\end{itemize}
Ograniczenia właściwe:
\begin{itemize}
    \item Każdego dnia suma trwania wszystkich zajęć jest mniejsza lub równa od czterech godzin:
    $$\left(\forall d \in [5]\right) \left(\sum_ {g \in G, c \in C : \delta_{g,c} = d} (\varepsilon_{g,c} - \sigma_{g,c}) \cdot x_{g,c} \leq 4\right)$$
    
    \item Dla każdego kursu wybieramy dokładnie jedną grupę:
    $$\left(\forall c \in C\right) \left(\sum_ {g \in G} x_{g,c} = 1\right)$$

    \item Student nie może brać udziału w dwóch różnych ćwiczeniach jednocześnie. Zatem jeśli dla dwóch zajęć danego dnia czas rozpoczęcia jednych z nich zawiera się między czasem rozpoczęcia i czasem zakończenia tych drugich to bierzemy co najwyżej jedne z nich:
    \small 
    $$
    \left(\forall c_1,c_2 \in C\right) 
    \left(\forall g_1,g_2 \in G\right) 
    \left(
    (c_1,g_1) \neq (c_2,g_2) \land 
    \delta_{g_1,c_1} = \delta_{g_2,c_2} \land 
    \sigma_{g_2, c_2} \in [ \sigma_{g_1, c_1}, \varepsilon_{g_1, c_1}] 
    \Rightarrow
    x_{g_1,c_1} + x_{g_2,c_2} \leq 1 
    \right)$$

    \item Podobnie ćwiczenia nie mogą zaczynać się w trakcie treningu:
    \small 
    $$
    \left(\forall c \in C\right) 
    \left(\forall g \in G\right) 
    \left(\forall p \in P \right) 
    \left(
    \delta_{g,c} = \delta^{PE}_{p} \land 
    \sigma_{g,c} \in [ \sigma^{PE}_{p}, \varepsilon^{PE}_{p}] 
    \Rightarrow
    x_{g,c} + y_p \leq 1 
    \right)$$

    \item Ani trening w trakcie ćwiczeń:
    \small 
    $$
    \left(\forall c \in C\right) 
    \left(\forall g \in G\right) 
    \left(\forall p \in P \right) 
    \left(
    \delta_{g,c} = \delta^{PE}_{p} \land 
    \sigma^{PE}_{p} \in [ \sigma_{g,c}, \varepsilon_{g,c}] 
    \Rightarrow
    x_{g,c} + y_p \leq 1 
    \right)$$

    \item Każdego dnia student musi mieć godzinę wolnego czasu między 12:00 i 14:00. Zatem suma czasu spędzonego na treningach i ćwiczeniach w tym okienku musi być mniejszy lub równy od 1:
    \begin{align*}
     (\forall d \in [5]) ( 
&\sum_ {g \in G, c \in C : \delta_{g,c} = d \land \sigma_{g,c} <   12 \land \varepsilon_{g,c} \leq 14} (\varepsilon_{g,c} - 12)                 \cdot x_{g,c} + \\  
    &\sum_ {g \in G, c \in C : \delta_{g,c} = d \land \sigma_{g,c} \geq  12 \land \varepsilon_{g,c} \leq 14} (\varepsilon_{g,c}  - \sigma_{g,c}) \cdot x_{g,c} + \\
    &\sum_ {g \in G, c \in C : \delta_{g,c} = d \land \sigma_{g,c} \geq  12 \land \varepsilon_{g,c} >  14} (14 - \sigma_{g,c})               \cdot x_{g,c} +\\
    &\sum_ {p \in P             : \delta^{PE}{p} = d \land \sigma^{PE}_{p} <   12 \land \varepsilon^{PE}_{p} \leq 14} (\varepsilon^{PE}_{p} - 12)                 \cdot y_{p}   + \\
    &\sum_ {p \in P             : \delta^{PE}{p} = d \land \sigma^{PE}_{p} \geq  12 \land \varepsilon^{PE}_{p} \leq 14} (\varepsilon^{PE}_{p}  - \sigma^{PE}_{p}) \cdot y_{p}   + \\
    &\sum_ {p \in P             : \delta^{PE}{p} = d \land \sigma^{PE}_{p} \geq  12 \land \varepsilon^{PE}_{p} >  14} (14 - \sigma^{PE}_{p})               \cdot y_{p}   \leq 1)   
    \end{align*}

\item Student trenuje co najmniej raz w tygodniu:
$$\sum_{p \in P} y_{p} \geq 1$$

\end{itemize}

Dodatkowe warunki:
\begin{itemize}
    \item Brak zajęć w środy i w piątki:
    $$\left( \forall g \in G, c \in C \right) (\delta_{g,c} \in \{3,5\} \Rightarrow x_{g,c} = 0)$$
    \item Brak ćwiczeń o preferencji mniejszej od 5:
    $$\left( \forall g \in G, c \in C \right) (\pi_{g,c} < 5 \Rightarrow x_{g,c} = 0)$$
\end{itemize}

\subsection{Funkcja celu}

\subsection{Wyniki}


\begin{center}
\begin{tabular}{|c|c|c|}
\hline
\textbf{Przedział czasu} & \textbf{Zużycie zasobu $r_1$} & \textbf{Wykonywane zadania}\\
\hline
(0,50) & 9 & 1 \\
\hline
(50,54) & 4 & 4 \\
\hline
(54,96) & 15 & 3,4 \\
\hline
(96,109) & 28 & 2,3 \\
\hline
(109,116) & 17 & 2 \\
\hline
(109,143) & 24 & 2,6 \\
\hline
(143,144) & 20 & 5,6 \\
\hline
(144,159) & 27 & 5,6,7 \\
\hline
(159,173) & 20 & 5,6 \\
\hline
(173,175) & 13 & 5 \\
\hline
(173,237) & 17 & 8 \\
\hline
\end{tabular}
\end{center}

\end{document}
